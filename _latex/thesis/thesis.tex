% From plainTeX template CTUStyle
% Run:  pdfcsplain your-file

\input ctustyle
%\input pdfuni    % Uncomment this if you need accented PDFoutlines
%\input opmac-bib % Uncomment this for direct reading of .bib database files 

%\shortcitations
\sortcitations

\worktype [M/EN]
\faculty {F3}
\department {Department of Computer Science}
\title {Estimating Bicycle Route Attractivity from Image Data}
\titleCZ {cz Estimating Bicycle Route Attractivity from Image Data}
\author {Vít Růžička}
\authorinfo {ruzicvi3@fel.cvut.cz}
\date {July 2017} 
\studyinfo {
   Master Programme: Open Informatics\nl
   Branch of Study: Computer Graphics and Interaction
}

\supervisor { % Supervisor including degrees, department and address
  Ing. Jan Drchal, Ph.D.\nl
  Department of Computer Science
}

\abstractEN {
abs
\hfil\break
}
\abstractCZ {
tract
\hfil\break
}

\keywordsEN {
	a
}
\keywordsCZ {
	b
\hfil\break
}

\thanks{I}

\def \dnes {\number\day. \ifcase\month\or ledna\or \'unora\or
   b\v rezna\or dubna\or kv\v etna\or \v cervna\or \v cervence\or
   srpna\or z\'a\v r\'i\or \v r\'ijna\or listopadu\or
   prosince\fi \space\number\year
}
\declaration {
  de
}

%%%%% <--   % The place for your own macros is here.

\input mymacros

%\draft     % Uncomment this if the version of your document is working only.
%\linespacing=1.7  % uncomment this if you need more spaces between lines
                   % Warning: this works only when \draft is activated!
%\savetoner        % Turns off the lightBlue backround of tables and
                   % verbatims, only for \draft version.
%\blackwhite       % Use this if you need really Black+White thesis.
%\onesideprinting  % Use this if you really don't use duplex printing. 

\makefront
%\nonum\notoc\sec Contents
%\maketoc \outlines0

% THESIS MAIN SECTION

\chap Introduction
%===============================

A


\chap Research
%===============================

B

\sec Planning
%===================

Planning

\secc Data collection
%------------------

Data collection


\sec History of Convolutional Neural Networks 
%===================

History of Convolutional Neural Networks


\secc ImageNet dataset
%------------------

Data collection

\secc ImageNet Large Scale Visual Recognition Competition (ILSVRC)
%------------------

Data collection

\seccc{AlexNet, CNN using huge datasets}1
%------------------

A

\seccc{VGG16, VGG19, going deeper}2
%------------------

A

\seccc{ResNet, recurrent connections and residual learning}3
%------------------

A

\seccc{Ensemble models}4
%------------------

A

\secc Feature transfer
%------------------

B

\secc Common structures
%------------------

B

\chap The Task
%===============================

\sec Route planning for bicycles 
%===================

A

\sec Available data and collection 
%===================

\secc Initial dataset
%------------------

I

\secc Street View
%------------------

I

\seccc{Downloading Street View images}1
%------------------

I

\seccc{Data augmentation introduction}2
%------------------

I

\secc Neighborhood in Open Street Map data
%------------------

I

\seccc{OSM neighborhood vector}1
%------------------

I

\seccc{Radius choice}2
%------------------

I

\seccc{Data transformation}3
%------------------

I

\chap The Method
%===============================

M

\sec Building blocks 
%===================

A

\secc Model abstraction
%------------------

A

\secc Fully-connected layers
%------------------

A

\secc Convolutional layers
%------------------

A

\secc Pooling layers
%------------------

A

\secc Dropout layers
%------------------

A

\sec Open Street Map neighborhood vector model 
%===================

A

\sec Street View images model 
%===================

A

\secc Model architecture
%------------------

I

\seccc{Base model}1
%------------------

I

\seccc{Custom top model}2
%------------------

I

\seccc{The final architecture}3
%------------------

I

\sec Mixed model 
%===================

A

\sec Data Augmentation 
%===================

A

\sec Model Training 
%===================

A

\secc Dataset split into validation and training data 
%------------------

A

\secc Training setting 
%------------------

A

\secc Specific setting for models using images 
%------------------

A

\secc Feature cooking 
%------------------

A

\sec Model evaluation 
%===================

A

\secc K-fold cross validation 
%------------------

A

\sec Frameworks and projects 
%===================

A

\secc Keras
%------------------

A

\secc Metacentrum project
%------------------

A

\chap The Implementation
%===============================


\sec Project overview 
%===================

A

\sec Experiment running 
%===================

A

\sec Downloader functionality 
%===================

A

\sec OSM Marker 
%===================

A

\sec Datasets and DatasetHandler 
%===================

A

\secc Dataset statistics
%------------------

C

\secc DatasetHandler structure
%------------------

D

\sec Models and ModelHandler 
%===================

A


\secc Model description in Keras
%------------------

C

\secc ModelHandler structure
%------------------

D

\sec Settings structure 
%===================

A

\sec Training 
%===================

A

\sec Testing 
%===================

A

\sec Reporting and folder structure 
%===================

A

\sec Metacentrum scripting 
%===================

A

\chap Results
%===============================

C

\chap Discussion
%===============================

B

\chap Conclusions
%===============================

V




\bye
