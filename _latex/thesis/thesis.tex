% From plainTeX template CTUStyle
% Run:  pdfcsplain your-file

\input eplain

\input ctustyle
%\input pdfuni    % Uncomment this if you need accented PDFoutlines
%\input opmac-bib % Uncomment this for direct reading of .bib database files 
%\shortcitations
\sortcitations

\worktype [M/EN]
\faculty {F3}
\department {Department of Computer Science}
\title {Estimating Bicycle Route Attractivity from Image Data}
\titleCZ {Odhadování atraktivity cyklistických tras z obrazových dat}
\author {Vít Růžička}
\authorinfo {ruzicvi3@fel.cvut.cz}
\date {July 2017} 
\studyinfo {
   Master Programme: Open Informatics\nl
   Branch of Study: Computer Graphics and Interaction
}

\supervisor { % Supervisor including degrees, department and address
  Ing. Jan Drchal, Ph.D.\nl
  Department of Computer Science
}

\abstractEN {
This master thesis focuses on practical application of Convolutional Neural Network models on the task of road labeling with bike attractivity score. We start with an abstraction of real world locations into nodes and scored edges in partially annotated dataset. We enhance information available about each edge with photographic data from Google Street View service and with additional neighborhood information from Open Street Map database. We teach a model on this enhanced dataset and experiment with ImageNet Large Scale Visual Recognition Competition. We try different dataset enhancing techniques as well as various model architectures to improve road scoring. We also make use of transfer learning to use features from a task with rich dataset of ImageNet into our task with smaller number of images, to prevent model overfitting.
\hfil\break
}
\abstractCZ {
Tato diplomová práce se zaměřuje na praktické použití konvolučních neuronových sítí na úloze ohodnocování jednotlivých úseků zvolené cesty pro cyklistu. Používáme lokality reálného světa abstrahované do struktury bodů a ohodnocených hran v částečně anotovaném datasetu. Tato původní data obohatíme o fotografickou informaci z lokace pomocí služby Google Street View a o vektorovou informaci objektů v blízkém sousedství z Open Street Maps databáze. Trénujeme model inspirovaný pokrokem v oblasti Počítačového vidění a moderními technikami použitými v soutěži  ImageNet Large Scale Visual Recognition Competition. Experimentujeme s různými metodami rozšiřování datasetu a s různými architekturami modelů ve snaze se co nejpřesněji přiblížit původnímu skórování. Používáme též metody přenosu příznaků z úlohy s dostatečně bohatým datasetem ImageNet na úlohu s menším množstvím obrázků, abychom předešli přeučování modelu.
\hfil\break
}

\keywordsEN {
Convolutional neural networks, planning, bicycle routing, machine learning, computer vision, object recognition, feature transfer, ImageNet, Google Street View
}
\keywordsCZ {
Konvoluční neuronové sítě, plánování, strojové učení, počítačové vidění, identifikace obrazových dat, přenos příznaků, ImageNet, Google Street View
\hfil\break
}


\thanks{
I would like to thank my supervisor Ing. Jan Drchal, Ph.D. for his help and leadership of this thesis and my alma mater Czech Technical University in Prague for giving me education.
I would further like to thank the Indian Institute of Technology Madras \inspic iitm.pdf and professor N. S. Narayanaswamy for his help with the research part of my thesis when I was on my study abroad in India.
I would also like to thank Hosei University \inspic jap_hu.pdf \ and professor Masami Iwatsuki for providing me with facilities to work on this thesis during my study abroad in Japan.
 \hfil\break
 \hfil\break

Access to computing and storage facilities owned by parties and projects contributing to the National Grid Infrastructure MetaCentrum provided under the programme “Projects of Large Research, Development, and Innovations Infrastructures” (CESNET LM2015042), is greatly appreciated.
}

\def \dnes {\number\day. \ifcase\month\or ledna\or \'unora\or
   b\v rezna\or dubna\or kv\v etna\or \v cervna\or \v cervence\or
   srpna\or z\'a\v r\'i\or \v r\'ijna\or listopadu\or
   prosince\fi \space\number\year
}

\declaration {
  I declare that I worked out the presented thesis independently
  and I quoted all used sources of information in accord with
  Methodical instructions about ethical principles for writing
  academic thesis.

   \signature % makes dots

  Vít Růžička

  \hfil\break 
  \indent Prague, \today

 \hfil\break
 \hfil\break
 \hfil\break
 \hfil\break
 \hfil\break
 \hfil\break
 \hfil\break
 \hfil\break
 \hfil\break
 \hfil\break
  \indent Prohlašuji, že jsem předloženou práci vypracoval samostatně,
  a~že jsem uvedl veškeré použité informační zdroje v~souladu
  s~Metodickým pokynem o~dodržování etických principů při přípravě
  vysokoškolských závěrečných prací.

   \signature % makes dots

  Vít Růžička

  \hfil\break 
  \indent V Praze, \dnes %2. května, 2014 
}


%%%%% <--   % The place for your own macros is here.
\input mymacros

%\draft     % Uncomment this if the version of your document is working only.
%\linespacing=1.7  % uncomment this if you need more spaces between lines
                   % Warning: this works only when \draft is activated!
%\savetoner        % Turns off the lightBlue backround of tables and
                   % verbatims, only for \draft version.
%\blackwhite       % Use this if you need really Black+White thesis.
%\onesideprinting  % Use this if you really don't use duplex printing. 

\makefront

% THESIS MAIN SECTION

\input chapters/ch00

\bibchap
\usebbl/c library

\input appendix/app01-abbrev
\input appendix/app02-overflow
\input appendix/app03-dataset
\input appendix/app05-cdcontent


\bye
