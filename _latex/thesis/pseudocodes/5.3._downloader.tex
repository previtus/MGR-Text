\documentclass[varwidth=38em]{standalone}
%\documentclass{article}
\usepackage{listings}
\usepackage{color}

\begin{document}

\lstset{language=Python, breaklines=true, deletekeywords={long}, 
morekeywords={
	def,for,in,if,then,else,while,do,assert,end,
	RunDownload, PrepSegments, GenListOfUrls, betweenPoints, DownloadUrlFilenameMap, url_retrieve_with_retry, SaveDataFile
	},
basicstyle=\footnotesize,       % the size of the fonts that are used for the code
numbers=left,                   % where to put the line-numbers
numberstyle=\footnotesize,      % the size of the fonts that are used for the line-numbers
stepnumber=1,                   % the step between two line-numbers. If it is 1 each line will be numbered
numbersep=5pt,                  % how far the line-numbers are from the code
backgroundcolor=\color{white},  % choose the background color. You must add \usepackage{color}
showspaces=false,               % show spaces adding particular underscores
showstringspaces=false,         % underline spaces within strings
showtabs=false,                 % show tabs within strings adding particular underscores
frame=single,           % adds a frame around the code
tabsize=2,          % sets default tabsize to 2 spaces
captionpos=b,           % sets the caption-position to bottom
breaklines=true,        % sets automatic line breaking
breakatwhitespace=false,    % sets if automatic breaks should only happen at whitespace
escapeinside={\%*}{*)},          % if you want to add a comment within your code
caption={RunDownloader},
}

\lstinputlisting[caption={RunDownloader -$>$ RunDownload}]{5.3._downloader.py}
\lstinputlisting[caption={RunDownloader -$>$ RunCheck}]{5.3._downloader_check.py}

\end{document} 